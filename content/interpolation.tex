%% contents/interpolation.tex
%% Copyright 2021-2022 Tom M. Ragonneau
%
% This work may be distributed and/or modified under the
% conditions of the LaTeX Project Public License, either version 1.3
% of this license or (at your option) any later version.
% The latest version of this license is in
%   http://www.latex-project.org/lppl.txt
% and version 1.3 or later is part of all distributions of LaTeX
% version 2005/12/01 or later.
%
% This work has the LPPL maintenance status `maintained'.
%
% The Current Maintainer of this work is Tom M. Ragonneau.
\chapter{Interpolating models for \glsfmtlong{dfo}}

We mentioned in \cref{ch:introduction} that model-based \gls{dfo} methods necessitate to approximate locally the functions involved in optimization problems by simple functions, referred to as \emph{models} or \emph{surrogates}.
These models are incorporated in subproblems that are in turn approximately minimized.
Examples of such functions commonly used in the literature are polynomials and \glspl{rbf}.
In this thesis, we focus on linear and quadratic polynomial functions.
We do not use higher-degree polynomials due to the following observations.
\begin{enumerate}
    \item The dimension of the space of polynomials in~$\R^n$ of degree at most~$k$ is~$\bigo(n^k)$.
    Therefore, evaluating an interpolating model from such a space necessitates~$\bigo(n^k)$ function evaluations (see \cref{sec:multivariate-interpolation}).
    Although we can reduce this number using underdetermined interpolation schemes (see \cref{sec:underdetermined-interpolation}), this amount of function evaluations is unacceptable in a \gls{dfo} context if~$k$ is too high,~$k \ge 3$ say.
    \item The method's subproblems are built upon these models. 
    Therefore, the more sophisticated the models are, the more difficult it is to solve approximately these subproblems.
    Since we are interested in numerical methods for \gls{dfo}, we want to evaluate approximate solutions to these subproblems fairly easily.
    \item As we mentioned hereinabove, subproblems are solved only approximately.
    They are usually approximately solved using methods that use first- and second-order derivatives of the models.
    Therefore, building high degree polynomial models is pointless if the methods do not use high-order information.
\end{enumerate}

We denote by~$\lpoly$ and~$\qpoly$ the spaces of respectively linear and quadratic polynomials on~$\R^n$.
We want to approximate a function~$\obj : \R^n \to \R$ with a model~$\objm$

\begin{itemize}
    \item Linear models~$\lpoly$, quadratic models~$\qpoly$, RBF models, \dots
    \item Definition of interpolation and regression.
    \item Let~$\mathcal{Y} \subseteq \R^n$ be the interpolation set.
    \item Dimensions of~$\lpoly$ and~$\qpoly$.
    \item Because of the dimensions of these space, we focus on interpolating models.
\end{itemize}

\section{Elementary concepts of multivariate interpolation}
\label{sec:multivariate-interpolation}

\begin{itemize}
    \item What is a simplex gradient? (needed in the introduction).
    \item Bases of~$\lpoly$ and~$\qpoly$.
    \item Poised set for interpolation.
    \item Characterization of the poisedness with bases.
\end{itemize}

\section{Overview of the Lagrange polynomials}

\begin{itemize}
    \item Definition of the Lagrange polynomials.
    \item They are well-defined if the interpolation set is poised.
    \item The Lagrange polynomials form a basis.
    \item Lebesgue constant and extension by~\cite{Ciarlet_Raviart_1972}.
    \item Interpretation of the absolute value of Lagrange polynomials using volumes of simplices.
\end{itemize}

\section{Poisedness of interpolation sets}
\label{sec:poisedness}

\subsection{Measuring the well poisedness of interpolation sets}

\begin{itemize}
    \item Definition of the~$\Lambda$-poisedness.
    \item The~$\Lambda$-poisedness measures the quality of the interpolation set.
    \item In practice we do not use it.
\end{itemize}

\subsection{Relationship with the conditioning of the interpolation system}

\begin{itemize}
    \item We consider here only the natural basis.
    \item The condition number is bounded by something that depends on~$\Lambda$.
\end{itemize}

\section{Underdetermined interpolation systems}
\label{sec:underdetermined-interpolation}

\begin{itemize}
    \item We focus on quadratic polynomials.
    \item Building a quadratic interpolating model necessitates~$\mathcal{O}(n^2)$ function evaluations.
    \item We want to reduce this number to~$\mathcal{O}(n)$.
    \item Freedom is bequeathed by minimizing a functional that reflects the regularity of the models.
    \item Two examples are given hereafter.
\end{itemize}

\subsection{Least Frobenius norm quadratic models}

\begin{itemize}
    \item Factorization of the symmetric Broyden matrix.
    \item Updating the symmetric Broyden matrix.
    \item Decompositions of the Hessian matrices of the quadratic models.
\end{itemize}

\subsection{Quadratic models based on symmetric Broyden updates}

\begin{itemize}
    \item Explain the name.
\end{itemize}
