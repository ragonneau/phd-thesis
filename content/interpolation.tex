%% contents/interpolation.tex
%% Copyright 2021-2022 Tom M. Ragonneau
%
% This work may be distributed and/or modified under the
% conditions of the LaTeX Project Public License, either version 1.3
% of this license or (at your option) any later version.
% The latest version of this license is in
%   http://www.latex-project.org/lppl.txt
% and version 1.3 or later is part of all distributions of LaTeX
% version 2005/12/01 or later.
%
% This work has the LPPL maintenance status `maintained'.
%
% The Current Maintainer of this work is Tom M. Ragonneau.
\chapter{Interpolation models for \glsfmtlong{dfo}}

\section{Introduction and motivation}

As mentioned in \cref{ch:introduction}, model-based \gls{dfo} methods necessitate to approximate locally the functions involved in optimization problems by simple functions, referred to as \emph{models} or \emph{surrogates}.
These models are used to construct subproblems that are in turn approximately minimized.
Examples of such functions commonly used in the literature are polynomials and \glspl{rbf}~\cite{Powell_2004a}.
In this thesis, we focus on linear and quadratic polynomial models.

Let~$\lpoly$ and~$\qpoly$ denote the spaces of polynomials on~$\R^n$ of degree at most one and two, respectively.
In a \gls{dfo} context, models from~$\lpoly$ or~$\qpoly$ are built for a real-valued function~$\obj$ without using derivatives.
This can be done by interpolation schemes based on function values.
More specifically, given a finite set of points~$\xpt \subseteq \R^n$, we construct a model~$\objm$ that interpolates the function~$\obj$ on~$\xpt$, i.e.,
\begin{equation}
    \label{eq:interpolation-conditions}
    \objm(y) = \obj(y), \quad \text{for~$y \in \xpt$}.
\end{equation}

The conditions~\cref{eq:interpolation-conditions} may be inconsistent.
In such a case, models can be built using regression schemes.
For example, a least-square regression model~$\objm$ minimizes
\begin{equation*}
    \sum_{\mathclap{y \in \xpt}} [\obj(y) - \objm(y)]^2.
\end{equation*}
Although there are successful methods that use regression models (see, e.g.,~\cite{Billups_Larson_Graf_2013,Conn_Scheinberg_Vicente_2008b}), the \gls{dfo} methods we present and develop in this thesis use interpolation models and ensure that the interpolation conditions are consistent and well-conditioned (see \cref{ch:pdfo,ch:cobyqa-introduction}).

It is also possible to use polynomials of degree higher than two.
However, we do not consider such models in this thesis due to the following observations.
\begin{enumerate}
    \item As shown in~\cite[thm.~2.5]{Wendland_2005}, the space of polynomials on~$\R^n$ of degree at most~$k$ has a dimension of
    \begin{equation*}
        \binom{n + k}{n} = \frac{1}{k!} \prod_{i = 1}^k (n + i) \ge \frac{n^k}{k!}.
    \end{equation*}
    Therefore, to determine a model from this space merely by the interpolation conditions~\cref{eq:interpolation-conditions}, we need in general~$\bigo(n^k)$ function values.
    This amount is unacceptable in a \gls{dfo} context unless~$k$ is small.
    It is possible to reduce this number with underdetermined interpolation, which is used by several optimization methods for~$k \le 2$ (see \cref{sec:underdetermined-interpolation}), including the method \gls{cobyqa} developed in this thesis (see \cref{ch:cobyqa-introduction}).
    Using underdetermined interpolation models with~$k \ge 3$ is out of the scope of this thesis, although it is an interesting research direction.
    \item The \gls{dfo} methods need to solve approximately subproblems (e.g., trust-region subproblems) built upon these models.
    Sophisticated models usually lead to complicated subproblems to solve.
    On the other hand, even with models that are not quadratic, practical algorithms normally solve the subproblems based on first- or second-order approximations of the models, e.g., calculate the approximate Cauchy point discussed in~\cite[\S~6.3.3]{Conn_Gould_Toint_2000}.
    Therefore, building polynomial models of degree higher than two may not be necessary.
\end{enumerate}

Although we do not study \gls{rbf} models, we mention that there also exist many \gls{dfo} methods based on these models.
Examples of such methods include \gls{orbit}~\cite{Wild_Regis_Shoemaker_2008}, \gls{conorbit}~\cite{Regis_Wild_2017}, and \gls{boosters}~\cite{Oeuvray_Bierlaire_2009}.

\section{Elementary concepts of multivariate interpolation}
\label{sec:multivariate-interpolation}

Before studying properties of multivariate interpolation, we must introduce the following notion of poisedness, which defines uniquely interpolants.

\begin{definition}
    We say that the set~$\xpt$ is \emph{poised} if the interpolation system~\cref{eq:interpolation-conditions} is consistent and has a unique solution.
\end{definition}

Let of first consider the problem of finding a linear model~$\objm \in \lpoly$ satisfying the interpolation system~\cref{eq:interpolation-conditions} whenever~$\card \xpt = \dim \lpoly = n + 1$.
In the natural basis of~$\lpoly$, the system~\cref{eq:interpolation-conditions} can be reformulated as
\begin{equation}
    \label{eq:linear-interpolation-conditions}
    \alpha_0 + \sum_{i = 1}^n \alpha_i y_i = \obj(y), \quad \text{for~$y \in \xpt$},
\end{equation}
where~$y_i$ denotes the~$i$th component of~$y \in \xpt$, and where~$\alpha_i \in \R$ for~$i \in \set{0, 1, \dots, n}$.
If~$\xpt$ is poised for linear interpolation, given~$\set{\alpha_0^{\ast}, \alpha_1^{\ast}, \dots, \alpha_n^{\ast}}$ the solution to~\cref{eq:linear-interpolation-conditions}, the linear model~$\objm$ is thus defined for~$x \in \R^n$ by
\begin{equation*}
    \objm(x) = \alpha_0^{\ast} + \sum_{i = 1}^n \alpha_i^{\ast} x_i,
\end{equation*}
and the vector~$(\alpha_i^{\ast})_{i = 1, 2, \dots, n}$ is referred to as the \emph{simplex gradient} of~$\obj$ for the interpolation set~$\xpt$.
Such models are used for instance by \gls{cobyla}~\cite{Powell_1994}, a \gls{dfo} method for nonlinearly-constrained optimization (see \cref{subsec:cobyla}).

The problem of finding a linear model~$\objm \in \qpoly$ satisfying the interpolation system~\cref{eq:interpolation-conditions} whenever~$\card \xpt = \dim \qpoly = (n + 1)(n + 2) / 2$ is very similar.
The interpolation conditions~\cref{eq:interpolation-conditions} can be reforumulated in the natural basis of~$\qpoly$, given by
\begin{equation*}
    \set{1, x_1, x_2, \dots, x_n, x_1^2 / 2, x_2^2 / 2, \dots, x_n^2 / 2, x_1 x_2, x_1 x_3, \dots, x_{n - 1} x_n},
\end{equation*}
where each element implicitely denotes the corresponding function of~$x \in \R^n$.

\begin{itemize}
    \item Bases of~$\lpoly$ and~$\qpoly$.
    \item Poised set for interpolation.
    \item Characterization of the poisedness with bases.
\end{itemize}

\section{Overview of the Lagrange polynomials}

\begin{itemize}
    \item Definition of the Lagrange polynomials.
    \item They are well-defined if the interpolation set is poised.
    \item The Lagrange polynomials form a basis.
    \item Lebesgue constant and extension by~\cite{Ciarlet_Raviart_1972}.
    \item Interpretation of the absolute value of Lagrange polynomials using volumes of simplices.
\end{itemize}

\section{Poisedness of interpolation sets}
\label{sec:poisedness}

\subsection{Measuring the well poisedness of interpolation sets}

\begin{itemize}
    \item Definition of the~$\Lambda$-poisedness.
    \item The~$\Lambda$-poisedness measures the quality of the interpolation set.
    \item In practice we do not use it.
\end{itemize}

\subsection{Relationship with the conditioning of the interpolation system}

\begin{itemize}
    \item We consider here only the natural basis.
    \item The condition number is bounded by something that depends on~$\Lambda$.
\end{itemize}

\section{Underdetermined interpolation systems}
\label{sec:underdetermined-interpolation}

\begin{itemize}
    \item We focus on quadratic polynomials.
    \item Building a quadratic interpolation model necessitates~$\mathcal{O}(n^2)$ function evaluations.
    \item We want to reduce this number to~$\mathcal{O}(n)$.
    \item Freedom is bequeathed by minimizing a functional that reflects the regularity of the models.
    \item Two examples are given hereafter.
    \item The new method we introduce in \cref{ch:cobyqa-introduction} also uses quadratic models obtained by underdetermined interpolation.
\end{itemize}

\subsection{Least Frobenius norm quadratic models}

\begin{itemize}
    \item Factorization of the symmetric Broyden matrix.
    \item Updating the symmetric Broyden matrix.
    \item Decompositions of the Hessian matrices of the quadratic models.
    \item \gls{mnh}~\cite{Wild_2008}
\end{itemize}

\subsection{Quadratic models based on symmetric Broyden updates}

\begin{itemize}
    \item Explain the name.
    \item \gls{newuoa}~\cite{Powell_2006}
\end{itemize}
